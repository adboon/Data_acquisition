\chapter{Deliverables}
\label{sec:deliverables}
Here an overview is given of the deliverables. First, the idea of the research is described, then concrete goals are discussed. After this, it is discussed what goals have been (partially) achieved. At the end a plan for the future is shown.


\section{Abstract ideas}
\label{sec:deliv_ideas}
SAMENVOEGEN ONDERSTAANDE STUKKEN TEKST \\
Literature study into green water. A lot of research into simulations has not yet been able to affordably and accurately simulate. Most previous research into the green water itself thus conducted with experiments, as discussed in chapter \ref{sec:literature}. These experiments are mostly fairly simplified representations of reality, as shown in paragraph \ref{sec:lit_exp_green_water}. This has given a lot of insight and five different types of green water events have been identified, discussed in \ref{sec:lit_event_types}. It also left questions around the occurrence of green water and various green water event types.
As most research has been experimental research, the problem has been scaled. However, during green water events air can be trapped. When this happens. the scaling laws are broken. This prevents the experimental data being used for large scales. Understanding the differences between the scaled and full scale results will make the experiments at model scale more useful.\par 

Firstly, long-running experiments will be conducted to obtain long-term data. The long-term data set will allow for the identification of extreme loading events and the frequency of occurrence, giving more insight
into the hydrodynamics of the green water problem. The data from these experiments will also be used as input for a probabilistic model, the research subject of WP-C. To do these long-running experiments, the existing flume tank will be adapted. The flumetank is 7 m long, 2.35 m wide, and 0.5 m tank with an inlet and outlet on each end to create a constant current. Adaptations will be made as a wedge-shaped plunging
a wavemaker and a parabolic perforated beach will be added. This test facility will allow for continuous experiments with waves and current to be conducted. It will be a unique facility with many possibilities for
future research. \par 

Scaling effects will also be researched as lack of knowledge leads to inaccurate results for green water when scaled experiments are used in research. This is problematic because scaled experiments are the main methodology of researching green water. The scaling laws are broken for, for instance, air entrapment during
these experiments. The largest green water loads are found when air is entrapped by the water. To research the effects of scaling, parts of the long-running experiments will be repeated at larger scales in the existing
towing tanks at TU Delft.



\section{Concrete goals}
\label{sec:deliv_goals}
To give a clear view of what the research is focused on, some concrete goals are stated below. Goals 1, 3, 4, 5, 8, 9 hold potential to lead to publications.

%Important is to note that these goals are set so as to be able to make long term plans or decisions all in line with the same goal, and not necessarily as a bar that needs to be passed.


\begin{enumerate}
	\item The flumetank will be adapted to become a wave-current tank and the test facility will be validated by recreating experimental data from the towing tank
	\item Large data sets will be generated with a model in irregular waves running for 40 hours
	\item From the large data set correlations between measured parameters and green water event types will be found
	\item The difference between various green water event types will be quantitatively formulated
	\item Using the large data sets the stochastics of green water event types will be quantified
	\item Other possible research using the wave-current tank will be started (e.g. slamming, different models, different sea states) 
	\item Short time series where green water and air entrapment occur will be repeated at different scales, keeping the other parameters constant
	\item The effect of scaling on the resulting pressures of the green water event will be evaluated
	\item With the knowledge of the effect of scaling, a procedure to minimize the effect or effectively compensate for it is set up
\end{enumerate}

\section{Achieved goals}
\label{sec:deliv_achieved_goals}
From the above, concrete goals, goal 1 has been PARTIALLY????? achieved. In figures HERE FIGURES OF THE ADDITIONS TO THE FLUMETANK the parts that have been physically build are shown. \par 
Work on fulfilling goals 2 and 3 is started. This is shown in the form of paper setups in appendixes \ref{sec:setup1} and \ref{sec:setup2}.

\section{Planning for coming years}
\label{sec:deliv_what_to_come}



\subsection{Data Management Plan}
\label{sec:deliv_DMP}
