\chapter{Setup paper 2: Using large sets of experimental data to investigate green water}
\label{sec:setup2}
\section*{Abstract}
Green water is a non-linear and strongly complex problem. Green water events are also extreme events, which means that they do not occur often. Long running experimental research is be conducted to collect different occurrences of green water. \textit{Rest will have to be written once data is collected and we know what we can get out of it}
\section{Introduction}
\label{sec:setup2_intro}
Ships and offshore structures are out in the ocean to transport the many goods we send around the world. pump up oil, or place wind turbines. Waves impact these structures, causing large loads. Green water is one of these impact types.
Green water is water that impacts on deck or superstructures. It can lead to huge impact pressures. As \citet{Buchner2002} stated, green water is a non-linear and strongly complex problem. Green water events are also extreme events, which means that they do not occur often. \par 
Previous research into green water can be categorized into three groups: approximations, numerical and experimental \cite{ISSC2012}. 
Approximations are empirical and analytical methods. They are simple and fast but less accurate. They consist mostly of predicting freeboard exceedance, as most greenwater events occur when the freeboard is exceeded. ADD A PIECE ABOUT IDENTIFICATION OF DIFFERENT TYPES OF GREEN WATER EVENTS BY GRECO
Prediction of exceedence of the freeboard is complex due to the many dependencies \cite{Lee2020}. These are for instance wave steepness or structure motions \cite{Greco2001,Buchner2002}. These parameters also introduce nonlinearities \cite{Faltinsen2002}. Researchers have made probability distributions of the free-board exceedance based on empirical parameters. Examples are \citet{Buchner2002}, \citet{Cox2001} and \citet{Soares2005}.
Numerical methods are more accurate than these approximation methods. The problem with numerical simulations is that they take excessive computational time \cite{Soares2015}.  
Experiments for green water events are affordable and the most accurate method or researching green water.  \par
An overview of literature discussing experiments is created. This is shown in table \ref{tab:setup2_all_exp_greenwater}. It shows 24 papers and the structure type used in the research, as well as the flow. Most experiments are conducted for a simplified model and in regular waves. Only 3 of the 24 papers take current velocity into account. All the research discussed was conducted with limited testing times, except for the research by \citet{Hamoudi1998} which used measurements from a full scale vessel. \par 
\begin{table}[]
	\caption{Overview of previous experimental green water research}
	\begin{tabular}{ll|cc}
		\multicolumn{1}{l}{\textbf{Year}}      & \multicolumn{1}{l|}{\textbf{Research}}  & \multicolumn{1}{c}{\textbf{Structure type}} & \multicolumn{1}{c}{\textbf{Wave and flow type}}  \\ \hline 
		1995                                   &  \citet{Buchner1995}                            & Ship                                        & JONSWAP + flow                                 \\
		1998                                   & \citet{Hamoudi1998}                             & Ship                                        & JONSWAP + flow                                \\
		2000                                   & \citet{Ersdal2000}                             & Ship                                        & Irregular                            \\
		2001                                   & \citet{Stansberg2001}                           & Ship                                        & Irregular                              \\
		2001                                   & \citet{Cox2001}                               & Box above water                                        & JONSWAP                                \\
		2001                                   & \citet{Ogawa2001}                               & Ship                                        & Regular \& Irregular + flow                  \\
		2002                                   & \citet{Faltinsen2002}                         & Box                                         & Regular                               \\
		2003                                   & \citet{Ogawa2003}                               & Ship                                        & Regular \& Irregular                     \\
		2003                                   & \citet{Mori2003}                                & Box above water                                         & JONSWAP                               \\
		2004                                   & \citet{Greco2004}                               & Ship                                        & Breaking                                 \\
		2005                                   & \citet{Soares2005}                              & Ship                                        & JONSWAP                               \\
		2005                                   & \citet{Greco2005}                               & Box                                         & Regular                                 \\
		2007                                   & \citet{Greco2007}                               & Box                                         & Regular                                  \\
		2012                                   & \citet{Greco2012}                               & Ship                                        & Regular                                \\
		2012                                   & \citet{Lee2012}                                & Box                                         & Regular                                 \\
		2012                                   & \citet{Ariyarathne2012}                         & Box                                         & Breaking                                \\
		2013                                   & \citet{Liut2013}                                & Ship                                        & Breaking-dam                              \\
		2015                                   & \citet{Song2015}                                & Box above water                                         & Breaking                               \\
		2017                                   & \citet{Abdussamie2017}                          & TLP                                         & Irregular                               \\
		2017                                   & \citet{Scharnke2017}                            & Box above water                                         & Breaking                                \\
		2018                                   & \citet{Chuang2018}                              & Box                                         & Random                               \\
		2020                                   & \citet{Lee2020}                                 & Box                                         & Regular                                 \\
		2020                                   & \citet{Hernandez-Fontes2020}                    & Box                                         & Regular                               \\
		2020                                   & \citet{Hernandez-Fontes2020a}                   & Box                                         & Regular                                  
	\end{tabular}
	\label{tab:setup2_all_exp_greenwater}
\end{table}
 
This makes sense because long testing times in a controlled and measurable environment could be achieved by doing many separate runs in a towing tank and knitting them together, but this is very expensive. The other option to research green water is to make the occurence of green water very likely, making it possible to research it with a short testing time. This leaves many questions open around the statistics of green water, and also makes it likely that green water events that occur in situations where researchers won't expect them, won't be researched. To overcome this, long running experiments with many waves passing a model should be conducted.  \par 

%S: \\
%Nonlinear wave events, like green water and slamming can lead to problems. \\
%These event types are also very complex. \\
%Also, they are extreme wave events, meaning that they do not occur often \\
%Main methods of investigating: experiments, some simulations \\
%C: \\
%Simulations still very expensive \\
%Experiments done in towing tank or wave flume \\
%Either limited testing times or no forward velocities \\
%Long testing times are needed as these are rare events \\
%Can be overcome by knitting, but very expensive. Or by forcing occurence of green water but this will not give insight into the statistics and unexpected occurences. \\
%Large data sets needed for (name for what Sanne is doing)\\
%Long running experiments have to be conducted as the next steps in green water research \\
%

%A:\\
%Experiments with long testing times will be conducted \\
%Use new wave-current tank \\
%Green water experiments conducted\\

\section{Method}
\label{sec:s2_method}
Experiments with long testing times will be conducted in the wave-current tank. \\
Pre-processing of the large data sets (eliminate what is not good, what is good, label/categorize, simplify data (make it clear points)) \\ 
cluster points / do something else with the cleaned up data \\
%IMPORTANT 
be on the look out of passive handling of the data due to its size, or unsound statistical fiddling \cite{Fricke2015}. 
