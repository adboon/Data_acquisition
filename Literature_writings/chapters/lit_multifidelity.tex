These can be categorized as simplified low fidelity models (e.g.: natural problem hierarchies, early stopping criteria, course-grid approximations), projection-based low fidelity models (e.g.: proper orthogonal decomposition, reduced basis model, Krylov subspace method), and data-fit low fidelity models (e.g.: interpolation/regression, kriging, support vector machines (SVM)). \cite{Peherstorfer2018}. 
\subsection{Model management strategy}
\label{sec:lit_mfm_mms}
When working with multiple models, for instance multiple low fidelity models, a model management strategy is needed. A model management strategy distributes work among the models while providing theoretical guarantees that establish the accuracy and/or convergence of the outer-loop result \cite{Peherstorfer2018}.


\subsection{Surrogate modelling}
\label{sec:lit_mfm_sm}
An available method of combining fidelities is the multi-fidelity surrogate model in which fidelities are combined inside a surrogate model. A surrogate model is constructed using the
information of multiple models with different levels of accuracy. These models can also be surrogate models by themselves. Multi-fidelity surrogate model construction in an multi-fidelity model is optional and it can be done by using a deterministic or a non-deterministic method. After its construction it may be treated as a multi-fidelity model. \cite{Fernandez-Godino2016}. This can also be done using only low fidelity data, resulting in a low fidelity model.

\subsubsection{Deterministic modelling}
\label{sec:lit_mfm_dm}
With deterministic multi-fidelity modelling it is assumed that a function is known. This method has the advantage that within this function the physics of the system can be captured, making a multi-fidelity model based on the function physical. However, such a function should then be known. 

\subsubsection{Non-deterministic modelling}
\label{sec:lit_mfm_ndm}
For a non-deterministic modelling method there is no formula to build on and thus a statistical inference to treat the parameters uncertainty is needed \cite{Fernandez-Godino2016}. A Bayesian multi-fidelity method can be applied. \par 
Gaussian process is a flexible, convenient and widely used class of distributions to model prior knowledge about our data \cite{Fernandez-Godino2016}. This Gaussian process regression is a machine learning method and should thus be trained. This can be done with high fidelity data \cite{Bonfiglio2018b}. This means that the quality of the high fidelity data will have significant influence on the accuracy of the multi-fidelity model.
