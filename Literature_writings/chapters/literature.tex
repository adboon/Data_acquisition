\chapter{SETUP: Greenwater types}
\begin{itemize}
	\item S:
	Green water is a thing.
	
	\item C:
	There are still questions about green water, and it is difficult to predict, and thus design for. 
	This is difficult because different types of events captured in one model.
	First, further insights into all events
	As stated, statistics needed.
	
	\item Q:
	Is there a way to improve prediction of green water by looking at the different types of impacts as separate phenomenon?
	
	What are the statistics for each of the green water events?
	
	What are the overall lifetime loadings of green water on a vessel?
	
	
	\item A:
	Going to do long running experiments
	With methods to identify the different event types
	
	?????????????
\end{itemize}
 
Ships and offshore structures are out in the ocean to transport the many goods we send around the world, pump up oil, or place wind turbines. Waves impact these structures, causing large loads. Green water is one of these impact types.
Green water is water that impacts on deck or superstructures. It can lead to huge impact pressures. As \citet{Buchner2002} stated, green water is a non-linear and strongly complex problem. Also as during green water events, free surface, air and water interact in a way that can lead to entrained and entrapped air \cite{VanDerEijk2020a}. It was found that with increased forward speed, the probability of green water increased \cite{Greco2012, Hamoudi1998}. Most studies focused on head waves, but green water can also occur on the side of a vessel.
%\begin{itemize}
%	\item - Ship sail in the ocean
%	\item - They encounter waves
%	\item - These waves impact on the ship
%	\item - When a wave impacts on the deck or superstructure of a ship it is called green water
%\end{itemize} 
%Multiple events can lead to water coming on board, an example is when an incoming wave significantly exceeds the freeboard \cite{Soares2015}. 
%\begin{itemize}
%	\item - Green water is nonlinear.
%	\item - If and how it occurs depends on many aspects. ......... found that, .............. found the dependency..............
%	[THIS SHOULD BE A PART OF LITERATURE WHERE AN OVERVIEW OF WHAT DEPENDENCIES WE KNOW, AND THE PROBLEMS, ARE GIVEN]
%	\item - Correlations are shown, but still complex
%	\item - With this knowledge, still difficult to predict
%	\item - Attempts to have prediction methods (summation with citations)
%%	\item - Mostly for dam-break event type
%%	\item - But many more types .........
%%	\item - (physics of the event)
%%	\item - (level of understanding)
%%	\item - (local vs global forces)
%%	\item - (max forces)
%%	\item - 
%	\item - Capturing all in one model, but they are different
%	\item - Because of this, no good prediction method yet
%	\item - Need to first understand all the events better
%	\item - As stated by (MANY), statistics needed
%\end{itemize}

\subsection{Green water event types}
\label{sec:lit_event_types}
Green water events have been classified into four main categories: dam-break, plunging, plunging dam-break combination and hammer-fist events \cite{Greco2007,Chuang2019,Zhang2019}. The impacts can be an isolated impact event, but multiple events can also follow each other up, influencing the loading \cite{Kendon2010}.  For dam-break scenarios, a wall of water is created around the deck as a consequence of large relative vertical motions between the ship and the water \cite{Buchner2002}. As this water exceeds the freeboard, water flows onto deck. The subsequent fluid motion on the deck resembles a wet dam-break flow \cite{Faltinsen2002}.  Researchers have developed a good understanding of the impact due to this type of event \cite{Ariyarathne2012}. 
\par 
However, plunging and hammer-fist type events lead to the most severe impacts \cite{Greco2007}. This is because they lead to impulsive pressures. Impulsive pressures have a high pressure rise time, in contrast to non-impulsive pressures which have a slow rise time \cite{Chuang2019}. The maximum instantaneous pressure is related to this pressure rise time \cite{Song2015}. 
\par 
In contrast to the dam-break event, the plunging wave is not the result of a ship interaction with steep waves causing run-up \cite{Chuang2019,Faltinsen2002}. Plunging green water events are thought to be caused by an (almost) breaking wave impacting on deck or superstructures \cite{Faltinsen2002}. As the wave breaks and overtops the structure, the flow becomes multi-phased and chaotic as an air pocket is formed \cite{Temarel2016}. This air pocket can lead to pressure oscillations and pressure peaks \cite{Lee2020}. Trapped air introduces a randomness which leads to variations in impact pressure \cite{Ariyarathne2012}. 
\par
A hybrid event type called a plunging dam-break event has also been identified. This type of event is the result of the interaction of a steep wave with the bow resulting in a wave breaking on deck. 
Plunging dam-breaking is the most common type of green water \cite{Greco2007, Greco2012}. With this type, air is also being trapped.  \par 
Hammer-fist type of events are, together with the plunging events, the most severe as they lead to the largest impacts. A hammer-fist impact happens when either a wave is focused locally in front of the edge of the deck, or when a strong wave impacts the vertical wall of the structure, pushing wave run-up onto the deck while maintaining a positive free-board \cite{Chuang2019}. The hammer-fist impact is connected to steep, non-breaking wave. Hammer-fist impacts are blunt water-deck impacts. For these type of events there is no apparent air entrapment near the bow \cite{Greco2007} 
\par 
At this moment no clear quantification of the different events could be found in the literature. The found research made the classifications of the events based of of visuals \cite{Greco2007, Chuang2019, Zhang2019}. Based on the discussed literature it is thought that for dam-break, a distributed non-impulsive impact is expected, for plunging a local impulsive impact, for dam-break plunging global pressure distribution with higher pressures at the edge and non-impulsive impact, and for the hammer-fist a global distributed, impulsive impact is expected. NOG GOED NADENKEN, DIT IDD ZO? ANDERE CLASSIFICATIE? BASEREN OP TIME TRACES VAN PRESSURES? MISS OOK IETS TOEVOEGEN OVER LUCHT INSLUITING EN OF DAT ALS CLASSIFICATIE KAN WORDEN GEBRUIKT. MISS JUIST WEL HOE HET WATER AAN BOORD KOMT GEBRUIKEN? IS DEZE MANIER VAN CLASSIFICEREN NIET MISS IETS WAT IK WIL DOEN DOOR MIDDEL VAN SVM OF DISCISION TREES? MAAR HOE GA IK ZE ANDERS ONDERSCHEIDEN? MISS JUIST WEL ONDERSCHEIDEN DOOR DE INKOMENDE GOLF EN DE BEWEGING VAN HET SCHIP DAT ER BIJ HOORT??? \par
MOGELIJKE METHODE: EERST VISUEEL ALLES CLASSIFICEREN. DAARNA NAGAAN WELKE MANIER HET OOK KAN (PRESSURE TRACES, SPEED OF WATERFRONT (from camera footage), ETC.)

\subsection{Prediction methods}
Tools to help predict loading due to green water have been developed. They can be categorized into three groups: approximations, numerical and experimental \cite{ISSC2012}. 
\par 
Numerical methods are more accurate than these approximation methods. The problem with numerical simulations is that they take excessive computational time \cite{Soares2015}.  
\par 
Experiments for green water events are still being carried out today, as it is an affordable and relatively accurate method.
\par 
Approximations are empirical and analytical methods. They are simple and fast but less accurate. They consist mostly of predicting freeboard exceedance, as most greenwater events occur when the freeboard is exceeded. 
Prediction of exceedence of the freeboard is complex due to the many dependencies \cite{Lee2020}. These are for instance wave steepness or structure motions \cite{Greco2001,Buchner2002}. These parameters also introduce nonlinearities \cite{Faltinsen2002}. Researchers have made probability distributions of the free-board exceedance based on empirical parameters. Examples are \citet{Buchner2002}, \citet{Cox2001} and \citet{Soares2005}. 



\par 
DIFFERENT EVENTS AND MORE UNDERSTANDING NEEDED, NEEDED THROUGH STATISTICS

As stated by \cite{Bullock2007,Bogaert2010,Bredmose2009,Peregrine2003}, statistics needed
% The impact and loading caused by green water depend on many variables, such as the motions of the structure and the steepness of the waves \cite{Greco2001, Buchner2002}. Wave steepness of incident waves and wave-body interaction cause important nonlinear effects \cite{Faltinsen2002}. Two types of pressure variations were identified: impulsive type with a rapid rate of pressure rise and non-impulsive type with a gradual pressure rise rate. \cite{Ariyarathne2012}. The impact can be an isolated impact event, but multiple events can also follow each other up, influencing the loadings \cite{Kendon2010}. 

\newpage 

\subsection{Experiments in literature}
\label{sec:lit_exp_green_water} 
Experiments are still the most reliable way to find green water loading. A database with experimental results was developed for validation of the simulations \cite{Lee2012}.  Almost exclusively FPSO type vessels or container ships, or simplified structures (step structures and cubic shapes) have been researched \cite{Chuang2019}. The focus on FPSO's is because of the equipment on deck, but it is found that green water loadings are significant and should also be considered while designing other ship types \cite{Kudupudi2019}. \par 
Experimental work has been conducted by \citet{Buchner2002} and \citet{Greco2001} leading to insight into green water loading, both focussed on FPSO type vessels. From the 2D study by \citet{Greco2001}, it is found that various aspects of the ship are of importance. From most to least influence on the loading: Freeboard, wave steepness, relative vertical motion, coupled flow between deck and outside, local flow at the bow, 3D effects, local design of deckhouse, stem angle, trim angle, hydro-elasticity during impact.  Experiments are conducted, focussing on the influence of the bow overhang for a multipurpose cargo ship. This too is a factor in the resulting amount of green waters \cite{Benmansour2016}. Also, an increase in green water was found with an increase of forward velocity \cite{Greco2012}. Experiments using bubble image velocimetry techniques found the full green water velocities. A correlation between velocity and impact pressure was identified \cite{Song2015}. \par  
The most common form of water overtopping onto the deck is the dam-break scenario. This can be modelled with the Saint Venant shallow water equation implemented by \citet{Ritter1892}. Experiments show that this dam-break model captures the green water events reasonably well \cite{Ryu2007, Chuang2018}. The relationship between the initial water depth of the dam-break and the freeboard exceedance remains unclear \cite{Chuang2019}. In addition to dam-break solutions, shallow water equations are also directly applied to simulate green water flows \cite{Greco2012, Liut2013}. Green water can occur because of a negative freeboard as discussed in paragraph \ref{sec:lit_green_water}. However, hammer-fist events can occur when there is a positive freeboard \cite{Greco2001}. Still, research is conducted into finding the relative wave elevation to help predict the occurrence of green water \cite{Buchner1995, Soares2005, Mori2003, Cox2001}. This has not resulted in a generalized method due to the complexity of wave-structure interactions and the non-linear sea states and dynamic structure responses \cite{Chuang2019}.

\subsubsection{Air entrapment}
\label{sec:lit_air_entrapment}
ADD A SUMMARY/INFORMATION FROM KIGER 2011 TO THIS TEXT (ALSO ABOUT NONDIMENSIONAL NUMBERS) "As a final closing comment to this discussion, we acknowledge that a mechanistic approach
from direct observation also has its limitations, particularly when there is the potential for the nonlinear interactions of multiple dynamic processes." \par 
%HIER MEER ECHT EEN INTRO STUKJE VOOR EEN MOGELIJK PAPER VAN MAKEN \par
When experiments are conducted, the problem is scaled down to fit inside test facilities. 
%The previously mentioned research also evaluates the scaling effects that influence experiments \cite{Song2015}. These are investigated as 
For green water experiments, scaling laws are violated for the effect of air entrapment. Air entrapment in scaled experiments can for instance lead to surface tension influencing the green water \cite{Greco2005}. 
%Air entrapment is of importance as it influences the impact pressure. 
A high level of aeration can increase both the force and impulse on a structure \cite{Ariyarathne2012}. 
\par 
Air entrapment is also relevant for coastal engineering \cite{Chan1994}, sloshing \cite{Lugni2006}, and slamming events \cite{Guzel2019}. The literature from these research areas involving air entrapment is used to look into air entrapment. Through experimental research on waves impacting a vertical wall, it is found that small amounts of entrapped air lead to a significant increase in impact pressure. A large amount of entrapped air has a damping effect on the pressures \cite{Hattori1994}. With computational simulations, slamming events where air and water mix are also researched. Pockets of bubbles can cushion the pressures, while for small bubbles in the water the pressure oscillations are more intense \cite{Sun2019}. Particle image velocimetry (PIV) combined with pressure sensors showed oscillating behaviour for air entrapment during sloshing \cite{Lugni2006}. The cushioning effect of air entrapment on the loading for a cylinder entering water is investigated numerically and experimentally \cite{Guzel2019}. Air entrapped for ocean waves impacting on sea walls is experimentally researched. It shows that air can be compressed to a pressure of several atmospheres and pressure shock-waves can lead to pressures comparable with those of the initial impact \cite{Bredmose2009}. Research to improve numerical simulations of air entrapment has also been conducted in recent years \cite{Sun2019, VanDerEijk2020}.

\subsection{Theories of Interest}
Waves on wall (sloshing), slamming (ship on wave) and green water (wave on ship), some similar physics, so when looking at one, look if you can use theories and research from the others. Mostly useful for air entrapment and scaling. 


\begin{itemize}
	\item Piston model of Bagnold 1939 (ELP3). Can partially correct for the compressibility bias (Lafeber 2012) 
	\item Wagner 1932 (wedge entering water) (ELP2)
	\item Rankine-Hugoniot 1870, 1887, 1889 (ELP1)
	\item Braeunig et al.’s (2009) "Phenomenological study of liquid impacts through 2D compressible two-fluid numerical simulations"
	\item L. Brosset \& J.M. Ghidaglia 2013
	\item Oumeraci, H. 1993, does what I want to do (classifying impact types on forces instead of visuals), but than for a wall
	\item 
	
\end{itemize}

\subsection{Simulations}
\label{sec:lit_sim_green_water}
Despite difficulties with the numerical simulations of green water, work continued. Work on the numerical simulation has been focussed on reducing the costs of calculations and increasing the quality. This research is also mostly conducted for FPSO type vessels or container ships, or simplified structures. 
Even though several numerical models have been developed to predict impact pressure, most of them are based on simplified assumptions such as inviscid and incompressible fluid, and single-phase flows \cite{Ariyarathne2012}. 
\par 
One of the simulation methods is numerical time-domain simulations based on an incompressible flow solver operating on unstructured grids. It shows good agreement with experimental results \cite{Lu2012}. An SPH method was used to simulate an extreme plunging wave impinging on the deck \cite{Soares2015} and to predict the fluid behaviour for green water \cite{LeTouze2010}. A hybrid CFD method involving linear seakeeping and nonlinear CFD analysis, both using 3D modelling of the hull, was introduced \cite{Kim2013}. Also, a three-step method (CFD-BEM-FEM) has been introduced to evaluate the loads due to green water on a container ship \cite{Kudupudi2019}. The Natural Element Method (NEM) employs a CIP-based method and a particle method to simulate strongly nonlinear wave-body interaction problems and is promising to be a valid alternative for green water simulations \cite{Hu2010}.
A CFD method with VOF-multiphase and SST-turbulence gives access to
high spatial resolution free surface position, water velocities and load distribution, phenomena usually not available from experiments \cite{Mandate2018}. Simulations with a multiphase-flow software based on a free-surface capturing method is used to evaluate green water for a Wigley hull. To reduce numerical diffusion at the free surface a solid-liquid-gas flow coupling model is developed by adopting Blend Reconstruction Interface Capturing Scheme (BRICS) \cite{He2017}.
\par 
From research into simulating green water loading with dam-break simulations it is found using a turbulence model, in this case k-\(\epsilon\) turbulence closure, gives the most accurate simulation results \cite{Khojasteh2020}. By comparing vertical loadings found with experiments to simple potential theory-based simulations and commercial CFD code it was found that potential code is adequate to find vertical loadings due to isolated impact events \cite{Kendon2010}. A potential theory-based engineering tool called KINEMA3 is developed to predict wave-induced impact loads on FPSOs in steep irregular waves, and for use in design load analysis \cite{Schiller2014}. A combination of the KINEMA3 and a CFD tool using the finite volume VoF (STARCCM+) is introduced. KINEMA3 is used to generate inlet conditions and STARCCM+ to model detailed flow on the deck \cite{Pakozdi2014}. Another way of combining a potential and CFD tool is by using a potential solver which finds the motions of a vessel and then using a CFD solver to find the green water loading \cite{Zhu2009}. A numerical approach using potential theory in the frequency domain to predict the relative wave elevation response for an FPSO can also be used \cite{Wang2017}.


\chapter{SETUP: New way of investigating nonlinear wave events}
S: \\
Nonlinear wave events, like green water and slamming can lead to problems. \\
These event types are also very complex. \\
Main methods of investigating: experiments, some simulations \\
C: \\
Simulations still very expensive \\
Experiments done in towing tank or wave flume \\
Either limited testing times or no forward velocities \\
Can be overcome by knitting, but very expensive \\
Back in the day no problem, because limit on data we can handle \\
Not anymore, now the testing facilities and expenses are limiting \\
Large data sets needed for (name for what Sanne is doing)\\
This limited size of data sets possibly also prevents experimental data to be used for machine learning \\
Need for cheap method of generating large data sets for nonlinear wave impacts \\
Q:\\
Experimental method needed that can cheaply generate data for loads of different nonlinear wave events \\
A:\\
Wave-current tank good option. Adapting flumetank by placing a wavemaker and beach on it\\
Does have limits (wave reflections, scale), but can give great insight into the stochastic of events like green water and slamming \\
\\


Such a wave-current tank created \\
Green water experiments conducted to see if it works \\
Validated with green water experiments in towing tank (same model) \\


First insights from big data (to show value of such a test facility)\\





