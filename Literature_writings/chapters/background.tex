\chapter{Introduction}
\label{chap:introduction}
Current design methods are severely limited when considering the challenges of designing safe and effective structural systems for (future) complex marine systems. Accounting for extreme loading situations these structures will face during their lifetime is a particularly difficult design problem. To address this challenge the "Multi-fidelity probabilistic design framework for complex marine structures" project is started. The project is made up of three parts and the research proposed in this document will fulfil one of the three parts within the project: development of reduced-order wave loading models to realistically describe lifetime loading scenarios on novel marine systems. This part is focused on extreme non-linear loading conditions \cite{ResearchProposal2020}. 
\par 
The choice is made to focus on initial green water loading from water entering at the bow. Through experimental research, lifetime loading scenarios will be investigated. To achieve this the TU Delft flume tank will be fitted with a wavemaker and a beach. From literature study, shown in paragraph \ref{sec:lit_air_entrapment}, it is found that there is a lack of knowledge for the influence of air entrapment on green water loading and inaccuracies that occur during experiments due to scaling. This will also be researched. Combined with the other two parts of the project, a multi-fidelity design framework will also be built. 
\par 
In this document chapter \ref{chap:literature} shows a general overview of the literature study conducted, chapter \ref{chap:objective} discusses the objective of the research in more detail and chapter \ref{chap:planning} shows the planning. Additionally to this research proposal, a PhD agreement is sent.
\nopagebreak